%\begin{document}
\section{Transformations}
\label{sec:transformations}
%
As we increase \(\beta\), it becomes increasingly unlikely for the sampler to
tunnel between different topological sectors, resulting in overall poor
performance.
%
In order to combat this effect, we have been focusing on better understanding
the mechanism by which our model is able to ``learn'' how to sample from
different topological sectors.
%
Specifically, we looked at how the effective energy, \(\mathcal{H} -
\log|\mathcal{J}|\) varies during a trajectory, as shown in
Fig.~\ref{fig:energies_fb,}.
%
We immediately notice that the energy tends to increase during the first-half
of the trajectory, before decreasing back towards its initial value.
%
It appears that by increasing the energy towards the middle of the trajectory,
the sampler is able to overcome the potential barriers seperating regions of
distinct topological charge.
%
In~\cite{Neal_2012} the authors introduced this idea, referred to as
``tempering during the trajectory'' and methods for manually producing a
similar effect are discussed.
%
In Fig.~\ref{fig:sinQ_ridgeplots}, we look at how the continuously-valued
topological charge, \(\sin\mathcal{Q} \equiv
\frac{1}{2\pi}\sum_{P}\sin{\phi_{P}}\)
%
% \begin{equation}
%   \sin\mathcal{Q} \equiv \frac{1}{2\pi}\sum_{P}\sin{\phi_{P}}
% \end{equation}
%
varies during a trajectory.
%
% In \cite{Neal_2012}, this idea is introduced and referred to as ``tempering
% during the trajectory'', and is implemented by manually introducing a scaling
% factor in th
% This idea has been studied previously~\cite{Neal_2012},
%
\begin{figure}[htpb]
  \centering
  \includegraphics[width=0.75\textwidth]{transformations/energy_dists_traj}
  % \caption{The rescaled energy \(\mathcal{H} -
  % \sum{\log|\mathcal{J}|}\)} %at the start (blue), middle (orange), and end
  %   %(green) of a trajectory during inference.}%
  \label{fig:energies_fb}
\end{figure}
%
\begin{figure}[htpb]
  \centering
  % \begin{subfigure}{0.66\textwidth}
  %   \centering
  %   \includegraphics[width=\textwidth]{transformations/energy_dists_traj}
  % \end{subfigure}
  % \vspace{0.1cm}
  \begin{subfigure}{0.33\textwidth}
    \centering
    \includegraphics[width=\textwidth]{transformations/ridgeplots/Hwf_ridgeplot}
    % \caption{Forward direction, \(d = +1\)}
  \end{subfigure}
  \begin{subfigure}{0.33\textwidth}
    \includegraphics[width=\textwidth]{transformations/ridgeplots/Hwb_ridgeplot}
    % \caption{Backward direction, \(d = -1\)}
  \end{subfigure}
  % \caption{Illustration of the effective energy \(\mathcal{H} -
  %   \sum{\log|\mathcal{J}|}\) at each leapfrog step during the trajectory.}%
  %   \label{fig:effective_energy_ridgeplots}
% \end{figure}
% %
% \begin{figure}[htpb]
%   \centering
  \begin{subfigure}{0.33\textwidth}
    \includegraphics[width=\textwidth]{transformations/ridgeplots/plaqsf_ridgeplot}
  \end{subfigure}
  \begin{subfigure}{0.33\textwidth}
    \includegraphics[width=\textwidth]{transformations/ridgeplots/plaqsb_ridgeplot}
  \end{subfigure}
    % \caption{Error in the average plaquette, \(\langle \phi_{P}\rangle\) for
    % the forward (left) and backward (right) directions.}
  %%%%%%%%%%%%%%%%%%%%%%%%%%%%%%%%%%%%%
  \begin{subfigure}{0.33\textwidth}
    \includegraphics[width=\textwidth]{transformations/ridgeplots/sinQf_ridgeplot}
  \end{subfigure}
  \begin{subfigure}{0.33\textwidth}
    \includegraphics[width=\textwidth]{transformations/ridgeplots/sinQb_ridgeplot}
  \end{subfigure}
  %%%%%%%%%%%%%%%%%%%%%%%%%%%%%%%%%%%%%
  % \caption{Illustration of the effective energy \(\mathcal{H} -
  %   \sum{\log|\mathcal{J}|}\) (top-row), and the continuously-valued
  %   topoological charge, \(\sin\mathcal{Q}\) at each leapfrog step during the
  %   trajectory.}%
  \label{fig:sinQ_ridgeplots}
\end{figure}
%
% \begin{figure}[htpb]
%   \centering
%   \begin{subfigure}{0.31\textwidth}
%     \includegraphics[width=\textwidth]{transformations/intQf_start_xrPlot}
%     \caption{Start, forward \(d=+1\).}
%   \end{subfigure}
%   \begin{subfigure}{0.31\textwidth}
%     \includegraphics[width=\textwidth]{transformations/intQf_mid_xrPlot}
%     \caption{Middle, forward \(d=+1\).}
%   \end{subfigure}
%   \begin{subfigure}{0.31\textwidth}
%     \includegraphics[width=\textwidth]{transformations/intQf_end_xrPlot}
%     \caption{End, forward \(d=+1\).}
%   \end{subfigure}
%   \begin{subfigure}{0.31\textwidth}
%     \includegraphics[width=\textwidth]{transformations/intQb_start_xrPlot}
%     \caption{Start, backward \(d=-1\).}
%   \end{subfigure}
%   \begin{subfigure}{0.31\textwidth}
%     \includegraphics[width=\textwidth]{transformations/intQb_mid_xrPlot}
%     \caption{Middle, backward \(d=-1\).}
%   \end{subfigure}
%   \begin{subfigure}{0.31\textwidth}
%     \includegraphics[width=\textwidth]{transformations/intQb_end_xrPlot}
%     \caption{End, backward \(d=-1\).}
%   \end{subfigure}
%   \caption{Illustration of the topological charge, \(\mathcal{Q}\) for each
%   configuration at the beginning, middle and end of a trajectory proceeding in
%   the forward \(d = +1\) (top row) and backward \(d = -1\) (bottom row)
%   direction.}
% \end{figure}
