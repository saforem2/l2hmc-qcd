\section{Introduction}%
\label{sec:l2hmc_intro}
We describe a new technique for performing Hamiltonian Monte-Carlo
(HMC) simulations called: `Learning to Hamiltonian Monte Carlo'
(L2HMC)~\cite{2017arXiv171109268L} which expands upon the traditional HMC by
using a generalized version of the leapfrog integrator that is parameterized by
weights in a neural network.
%
Hamiltonian Monte-Carlo improves upon the random-walk guess and check strategy
of generic MCMC by integrating Hamilton's equations along approximate
iso-probability contours of phase space.
%
In doing so, we are able to explore
the phase space more efficiently by taking larger steps between proposed
configurations while maintaining high acceptance rates.
%
In order to demonstrate the usefulness of this new approach, we use various
metrics for measuring the performance of the trained (L2HMC) sampler vs.\ the
generic HMC sampler.

First, we will look at applying this algorithm to a two-dimensional Gaussian
Mixture Model (GMM).
%
The GMM is a notoriously difficult distribution for HMC due to the vanishingly
small likelihood of the leapfrog integrator traversing the space between the
two modes.
%
Conversely, we see that through the use of a carefully chosen training
procedure, the trained L2HMC sampler is able to successfully discover the
existence of both modes, and mixes (`tunnels') between the two with ease. 
%
Additionally, we will observe that the trained L2HMC sampler mixes much faster
than the generic HMC sampler, as evidenced through their respective
autocorrelation spectra.

This ability to reduce autocorrelations is an important metric for measuring
the efficiency of a general MCMC algorithm, and is of great importance for
simulations in lattice gauge theory and lattice QCD.\@
%
Following this, we introduce the two-dimensional $U(1)$ lattice gauge theory
and describe important modifications to the algorithm that are of particular
relevance for lattice models.
%
Ongoing issues and potential areas for improvement are also discussed,
particularly within the context of high-performance computing and long-term
goals of the lattice QCD community.
%%%%%%%%%%%%%%%%%%%%%%%%%%%%%%%%%%%%%%%%%%%%%%%%%%%%%%%%%%%%%%%%%%%%%%%%%%%%%%%
