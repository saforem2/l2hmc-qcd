%\begin{document}
\section{Separate Networks}%
\label{sec:separate_networks}
% \subsection{Single Network}%
% \label{subsec:single_network}
%
Recall that a single leapfrog step (in the \emph{forward} \((d = +1)\)
direction) of the L2HMC algorithm consists of the following 4 updates:
% \vspace{-20pt}
\begin{enumerate}
  \item \(\mathrm{\textbf{vNet}}: \zeta_{1} \coloneqq (x, \partial_{x}U(x), t)
    \longrightarrow (S_{v}, T_{v}, Q_{v})\):
    \begin{equation}
      \vp = v \odot \exp{\left(\frac{\eps}{2}S_{v}(\zeta_{1})\right)}% 
             - \frac{\eps}{2}\left[\partial_{x}\,U(x)\odot \exp{\left(\eps
               Q_{v}(\zeta_{1})\right)}%
             + T_{v}(\zeta_{1})\right]
    \end{equation}
  \item \(\mathrm{\mathbf{xNet}}: \zeta_{2} \coloneqq 
    (x_{\bar{m}^{t}}, \vp, t) \longrightarrow (S_{x}, T_{x}, Q_{x})\):
    \begin{equation}
      \xp = x_{\bar{m}^{t}} + m^{t}\odot \left[x \odot \exp{\left(\eps
              S_{x}(\zeta_{2})\right)}%
            + \eps\left(\vp\odot\exp{\left(\eps Q_{x}(\zeta_{2})\right)} 
            + T_{x}(\zeta_{2})\right)\right]
    \end{equation}
  \item \(\mathrm{\mathbf{xNet}}: \zeta_{3} \coloneqq (x_{m^{t}}^{\prime}, \vp,
    t) \longrightarrow (S_{x}, T_{x}, Q_{x})\):
    \begin{equation}
      \xpp = x^{\prime}_{m^{t}} + \bar{m}^{t}\odot \left[\xp \odot \exp{\left(\eps
              S_{x}(\zeta_{3})\right)}%
            + \eps\left(\vp\odot\exp{\left(\eps Q_{x}(\zeta_{3})\right)}
            + T_{x}(\zeta_{3})\right)\right]
    \end{equation}
  \item \(\mathrm{\mathbf{vNet}}: \zeta_{4} \coloneqq (\xpp,
    \partial_{x}U(\xpp), t) \longrightarrow (S_{v}, T_{v}, Q_{v})\):
    \begin{equation}
      \vpp = \vp \odot \exp{\left(\frac{\eps}{2}S_{v}(\zeta_{4})\right)}%
              - \frac{\eps}{2}\left[\partial_{x}\,U(\xpp)\odot \exp{\left(\eps
                  Q_{v}(\zeta_{4})\right)}%
              + T_{v}(\zeta_{4})\right].
    \end{equation}
\end{enumerate}
%

A complete trajectory (of \(\Nlf\) leapfrog steps) is then obtained via
\(\Nlf\) successive applications of the above procedure, followed by a MH
accept/reject step.
%
Note that in this approach we use the \textbf{same two networks for all
leapfrog steps}, namely: \(\{\mathrm{\mathbf{xNet}}, \mathrm{\mathbf{vNet}}\}\),
which update \(\{x, v\}\) respectively.
%
% \subsection{Separate Networks}%
% \label{subsec:separate_networks}
%
This can be generalized by introducing \textbf{different networks for each
leapfrog step} (e.g.\@ the first LF step uses \(\{\xNet^{(1)},
\vNet^{(1)}\}\), the second uses \(\{\xNet^{(2)}, \vNet^{(2)}\}\), \(\ldots\), and
the final step uses \(\{\xNet^{(\Nlf)}, \vNet^{(\Nlf)}\}\)).
%
While this approach dramatically increases the number of trainable parameters,
it is also vastly more expressive and allows the model to learn independent
updates for each leapfrog step.
%

Explicitly, let \(\LBtheta^{(i)}: \xi \rightarrow \xip\) denote the operator which performs
the \(i^{th}\) (augmented) leapfrog step in the direction \(d\), and introduce
\(\Gamma_{\pm}^{(i)}: \left(v; \zeta_{v}\right) \rightarrow \vp\),
\(\Lambda_{\pm}^{(i)}: \left(x; \zeta_{x}\right)\rightarrow \xp\),
%
where \(\zeta_{v} = (x, \partial_{x}U(x), t)\), \(\zeta_{x} = (x, v, t)\).
% , and
% the functions \(\GammaPMi, \LambdaPMi\) update the momentum and position
% variables, respectively.
%
We can then write the full position and momentum updates in terms of these
functions
%
\begin{align}
  \Gamma_{\pm}^{(i)}\left(v; \zeta_{v}\right) % \coloneqq \vp =
    &= \begin{cases}%\,\,\,
      v \odot \exp{\left(\frac{\eps}{2}\Svi(\zeta_{v})\right)}% 
        - \frac{\eps}{2}\left[\partial_{x}\,U(x)\odot \exp{\left(\eps
            \Qvi(\zeta_{v})\right)}%
        + \Tvi(\zeta_{v})\right]
        &\quad (d = + 1)\\
      % -------------------
      % \vspace{0.75cm}%\,\,\,
      % -------------------
      \left\{v + \frac{\eps}{2}\left[\partial_{x}\,U(x)\odot
          \exp{\left(\eps \Qvi(\zeta_{1})\right)}%
            + \Tvi(\zeta_{1})\right]\right\} \odot
            \exp{\left(-\frac{\eps}{2}\Svi(\zeta_{1})\right)}
      &\quad (d = - 1)
    \end{cases}\\
  %%%%%%%%%%%%%%%%%
  % \vspace{2.5cm}
    &\quad \nonumber \\
  %%%%%%%%%%%%%%%%%
  \Lambda_{\pm}^{(i)}\left(x; \zeta_{x}\right) % \coloneqq \xp
    &= \begin{cases}%\,\,\,%
      x \odot \exp{\left(\eps
        \Sxi(\zeta_{x})\right)}%
        + \eps\left[\vp\odot\exp{\left(\eps \Qxi(\zeta_{x})\right)} 
        + \Txi(\zeta_{x})\right]%
      &\quad (d = + 1) \\
      % -------------------
      % \vspace{0.75cm}%\,\,\,
      % -------------------
      \left\{x - \eps{\left[\vp\odot\exp{\left(\eps \Qxi(\zeta_{x})\right)}%
          + \Txi(\zeta_{x})\right]}\right\}\odot \exp{\left(-\eps
          \Sxi(\zeta_{x})\right)}%
      &\quad (d = + 1)
  \end{cases}\\
\end{align}
%
We can write the \(i^{th}\) leapfrog step as:
%
\begin{enumerate}
  \item \(v \longrightarrow \vp = \Gamma_{\pm}^{(i)}(v; \zeta_{v})\)
  \item \(x \longrightarrow \xp = x_{\bar{m}^{t}} +
    \Lambda_{\pm}^{(i)}(x_{m^{t}}; \zeta_{x})\)
  \item \(\xp \longrightarrow \xpp = \xp_{\bar{m}^{t}} +
    \Lambda_{\pm}^{(i)}(\xp_{m^{t}}; \zeta_{x}^{\prime})\)
  \item \(\vp \longrightarrow \vpp = \Gamma_{\pm}^{(i)}(\vp; \zeta_{v}^{\prime\prime})\)
\end{enumerate}
%
and we can obtain a complete trajectory (of \(\Nlf\) leapfrog steps) by
succesively applying the above steps.
%
