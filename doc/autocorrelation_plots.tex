\subsection{Autocorelation of the topological charge \texorpdfstring{$Q$}{Q}}
%
For each of the experiments ran above, I also looked at the autocorrelation of the topological charge $Q$ vs. MD step.
%
These plots are also included below.
\begin{figure}[htpb]\label{fig:charge_autocorrelation_lf5}
  \includegraphics[width=0.49\textwidth]{autocorrelations/charge_autocorr_lf5_beta5.eps}
  \hfill
  \includegraphics[width=0.49\textwidth]{autocorrelations/charge_autocorr_lf5_beta6.eps}
  \caption{Autocorrelation of the topological charge vs. MD step for $\beta = 5.0$ (left) and $\beta = 6.0$ (right). In
    this example, each MD step consists of $N_{\mathrm{LF}} = 5$ leapfrog steps.}
\end{figure}
%
\begin{figure}[htpb]\label{fig:charge_autocorrelation_lf6}
  \includegraphics[width=0.49\textwidth]{autocorrelations/charge_autocorr_lf6_beta5.eps}
  \hfill
  \includegraphics[width=0.49\textwidth]{autocorrelations/charge_autocorr_lf6_beta6.eps}
  \caption{Autocorrelation of the topological charge vs. MD step for $\beta = 5.0$ (left) and $\beta = 6.0$ (right). In
    this example, each MD step consists of $N_{\mathrm{LF}} = 6$ leapfrog steps.}
\end{figure}
%
\begin{figure}[htpb]\label{fig:charge_autocorrelation_lf7}
  \includegraphics[width=0.49\textwidth]{autocorrelations/charge_autocorr_lf7_beta5.eps}
  \hfill
  \includegraphics[width=0.49\textwidth]{autocorrelations/charge_autocorr_lf7_beta6.eps}
  \caption{Autocorrelation of the topological charge vs. MD step for $\beta = 5.0$ (left) and $\beta = 6.0$ (right). In
    this example, each MD step consists of $N_{\mathrm{LF}} = 7$ leapfrog steps.}
\end{figure}
%
\begin{figure}[htpb]\label{fig:charge_autocorrelation_lf8}
  \includegraphics[width=0.49\textwidth]{autocorrelations/charge_autocorr_lf8_beta5.eps}
  \hfill
  \includegraphics[width=0.49\textwidth]{autocorrelations/charge_autocorr_lf8_beta6.eps}
  \caption{Autocorrelation of the topological charge vs. MD step for $\beta = 5.0$ and $\beta=6.0$. In this example,
    each MD step consists of $N_{\mathrm{LF}} = 8$ leapfrog steps.}
\end{figure}
%
\begin{figure}[htpb]\label{fig:charge_autocorrelation_lf9}
  \includegraphics[width=0.49\textwidth]{autocorrelations/charge_autocorr_lf9_beta5.eps}
  \hfill
  \includegraphics[width=0.49\textwidth]{autocorrelations/charge_autocorr_lf9_beta6.eps}
  \caption{Autocorrelation of the topological charge vs. MD step for $\beta = 5.0$ (left) and $\beta = 6.0$ (right). In
    this example, each MD step consists of $N_{\mathrm{LF}} = 9$ leapfrog steps.}
\end{figure}
%
\begin{figure}[htpb]\label{fig:charge_autocorrelation_lf10}
  \includegraphics[width=0.49\textwidth]{autocorrelations/charge_autocorr_lf10_beta5.eps}
  \hfill
  \includegraphics[width=0.49\textwidth]{autocorrelations/charge_autocorr_lf10_beta6.eps}
  \caption{Autocorrelation of the topological charge vs. MD step for $\beta = 5.0$ (left) and $\beta = 6.0$ (right). In
    this example, each MD step consists of $N_{\mathrm{LF}} = 10$ leapfrog steps.}
\end{figure}
%
\begin{figure}[htpb]\label{fig:charge_autocorrelation_lf15}
  \includegraphics[width=0.49\textwidth]{autocorrelations/charge_autocorr_lf15_beta5.eps}
  \hfill
  \includegraphics[width=0.49\textwidth]{autocorrelations/charge_autocorr_lf15_beta6.eps}
  \caption{Autocorrelation of the topological charge vs. MD step for $\beta = 5.0$ (left) and $\beta = 6.0$ (right). In
    this example, each MD step consists of $N_{\mathrm{LF}} = 15$ leapfrog steps.}
\end{figure}

\clearpage
