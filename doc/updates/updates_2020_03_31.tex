\clearpage
\section{Updates: 03/31/2020}%
\label{sec:updates_2020_03_31}
\subsection{Periodicity}%
\label{subsec:periodicity}
\begin{itemize}
  \item When using the angular representation, \(x \equiv \phi_{\mu}(k) \in [0,
    2\pi)\) of the link variables \(U_{\mu}(k) = e^{i\phi_{\mu}(k)}\in U(1)\),
    it seems like the main problem (possibly in addition to issues with
    reversibility) is that the output is discontinuous when the input moves
    between \(0\), and \(2\pi -\varepsilon\) (\(\varepsilon \ll 1\)).
    % the network is \emph{not} gauge
    % periodic, i.e. \(f(x) \neq
    % f(x + 2\pi)\).
  % \item Because of this, taking
  %   \begin{equation}
  %     x \longrightarrow x \pmod{2\pi}
  %   \end{equation}
  %   % violates reversibility since information about \(x\) is lost.
  \item Additionally, the network itself is not periodic, i.e.\ if \(x
    \longrightarrow x + n\pi\), the \(x\odot \exp(\varepsilon
    S_{x}(\zeta_{1}))\) term in the \(x\)-update becomes
    \begin{equation}
      x\odot \exp(\varepsilon S_{x}(\zeta_{1})) \longrightarrow (x +
      n\pi)\odot\exp(\varepsilon S_{x}(\zeta_{1}))
    \end{equation}
    and we're left with an additional \(n\pi \odot \exp(\varepsilon
    S_{x}(\zeta_{1}))\) term.
  \item We can avoid this issue by using the \(\vec{x} \equiv
    \left[\cos\phi_{\mu}(k), \sin\phi_{\mu}(k)\right]\) representation as input
    to the network, which then updates the \(\cos\) and \(\sin\) terms
    separately.
\end{itemize}
