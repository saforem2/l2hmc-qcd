\subsection{Metropolis-Hastings Accept/Reject}
Written in terms of these transformations, the augmented leapfrog operator
$\mathbf{L}_{\theta}$ consists of $M$ sequential applications of the
single-step leapfrog operator $\mathbf{L}_{\theta} \xi = \mathbf{L}_{\theta}(x,
v, d) = (x^{\prime\prime\times M}, v^{\prime\prime\times M}, d)$, followed by
the previously-defined momentum flip operator $\mathbf{F}$ which flips the
direction variable $d$, i.e.\ $\mathbf{F}\xi = (x, v, -d)$.
%
Using these, we can express a complete molecular dynamics update step as
$\mathbf{FL}_{\theta}\xi = \xip$, where now the Metropolis-Hastings acceptance
probability for this proposal is given by
%
\begin{equation}
    A(\mathbf{F}\mathbf{L} \xi | \xi) = \min\left(1,
        \frac{p(\mathbf{F}\mathbf{L}\xi)}{p(\xi)}\left|
        \frac{\partial\left[\mathbf{F}\mathbf{L}\xi\right]}
            {\partial\xi^{T}}\right|\right),
\end{equation}
%
Where $\left|\frac{\partial\left[\mathbf{F}\mathbf{L}\xi\right]}
{\partial\xi^{T}}\right|$ denotes the determinant of the Jacobian describing
the transformation.

In contrast to generic HMC where
$\left|\frac{\partial\left[\mathbf{F}\mathbf{L}\xi\right]}
{\partial\xi^{T}}\right| = 1$, we now have non-symplectic transformations
(i.e.\ non-volume preserving) and so we must explicitly account for the
determinant of the Jacobian.
%
These non-volume preserving transformations have the effect of deforming the
energy landscape, which, depending on the nature of the transformation, may
allow for the exploration of regions of space which were previously
inaccessible.
%
\newcommand{\energyA}{\includegraphics[width=0.4\textwidth]{energy_landscape/original_energy_landscape.pdf}}
\newcommand{\energyB}{\includegraphics[width=0.4\textwidth]{energy_landscape/modified_energy_landscape.pdf}}
%
\begin{figure}
  \centering 
  \Huge
  \parbox{\widthof{\energyA}}{\energyA} $\overset{\mathcal{J}}{\longrightarrow}$
  \parbox{\widthof{\energyB}}{\energyB} 
  \normalsize
  \caption{Example of how the determinant of the Jacobian can deform the energy landscape.}
\end{figure}

To simplify our notation, introduce an additional operator $\mathbf{R}$ that
re-samples the momentum and direction, e.g.\ given $\xi = (x, v, d)$,
$\mathbf{R}\,\xi = (x, v^{\prime}, d^{\prime})$ where $v^{\prime} \sim
\mathcal{N}(0, I)$, $d^{\prime} \sim \mathcal{U}\left(\{-1, 1\}\right)$.
%
A complete sampling step of our algorithm then consists of the following two
steps:
%
\begin{enumerate}
    \item $\xi^{\prime} = \mathbf{FL}_{\theta} \,\xi$ with probability
        $A(\mathbf{FL}_{\theta}\,\xi|\xi)$ %(Eq.~\ref{eq:metropolis_hastings}),
        otherwise $\xi^{\prime} = \xi$.
    \item $\xi^{\prime} = \mathbf{R}\,\xi$.
\end{enumerate}
%
Note however, that for MH to be well-defined, this deterministic operator must
be \emph{invertible} and \emph{have a tractable Jacobian} (i.e.\ we can compute
its determinant).
%
In order to make this operator invertible, we augment the state space $(x, v)$
into $(x, v, d)$, where $d \in \{-1, 1\}$ is drawn with equal probability and
represent the direction of the update.
%
All of the previous expressions for the augmented leapfrog updates represent
the forward ($d = 1$) direction.
%
We can derive the expressions for the backward direction ($d = -1$) by
reversing the order of the updates (i.e.\ $\vpp \rightarrow \vp$, then $\xpp
\rightarrow \xp$, followed by $\xp \rightarrow x$ and finally $\vp \rightarrow
v$).
%
For completeness, we include in Sec.~\ref{sec:lf_forward} and
Sec~\ref{sec:lf_backward} all of the equations (both forward and backward
directions) relevant for updating the variables of interest in our augmented
leapfrog sampler.
\section{Forward Direction \texorpdfstring{$(d = 1)$}{(d = 1)}:}%
\label{sec:lf_forward}
% \vspace{-20pt}
\begin{align}
  \label{eq:forward_update}
  \vp &= v \odot \exp{\left(\frac{\eps}{2}S_{v}(\zeta_{1})\right)}% 
        - \frac{\eps}{2}\left[\partial_{x}\,U(x)\odot \exp{\left(\eps
          Q_{v}(\zeta_{1})\right)}%
        + T_{v}(\zeta_{1})\right] \\
  \xp &= x_{\bar{m}^{t}} + m^{t}\odot \left[x \odot \exp{\left(\eps
    S_{x}(\zeta_{2})\right)}%
        + \eps\left(\vp\odot\exp{\left(\eps Q_{x}(\zeta_{2})\right)} 
          + T_{x}(\zeta_{2})\right)\right] \\
  \xpp &= x^{\prime}_{m^{t}} + \bar{m}^{t}\odot \left[\xp \odot \exp{\left(\eps
    S_{x}(\zeta_{3})\right)}%
        + \eps\left(\vp\odot\exp{\left(\eps Q_{x}(\zeta_{3})\right)} +
      T_{x}(\zeta_{3})\right)\right] \\
  \vpp &= \vp \odot \exp{\left(\frac{\eps}{2}S_{v}(\zeta_{4})\right)}%
        - \frac{\eps}{2}\left[\partial_{x}\,U(\xpp)\odot \exp{\left(\eps
          Q_{v}(\zeta_{4})\right)}%
          + T_{v}(\zeta_{4})\right]
\end{align} 
%
With $\zeta_{1} = (x, \partial_{x}\, U(x), t)$, $\zeta_{2} = (x_{\bar{m}^{t}},
v, t)$, $\zeta_{3} = (x^{\prime}_{m^{t}}, v, t)$, $\zeta_{4} = (\xpp,
\partial_{x}\, U(\xpp), t)$.
%%%%%%%%%%%%%%%%%%%%%%%%%%%%%%%%%%%%%%%%%%%%%%%%%%%%%%%%%%%%%%%%%%%%%%%%%%%%%%%
%
\section{Backward Direction \texorpdfstring{$(d = -1)$}{(d = -1)}:}%
\label{sec:lf_backward}
%
\begin{align}
  \label{eq:backward_update}
  v^{\prime} &= {\left\{v + \frac{\eps}{2}\left[\partial_{x}\,U(x)\odot
        \exp{\left(\eps Q_{v}(\zeta_{1})\right)}%
    + T_{v}(\zeta_{1})\right]\right\}} \odot
    \exp{\left(-\frac{\eps}{2}S_{v}(\zeta_{1})\right)} \\
  \xp &= x_{m^{t}} + \bar{m}^{t}\odot%
    {\left[x - \eps{\left(\exp{\left(\eps Q_{x}(\zeta_{2})\right)}\odot \vp%
    + T_{x}(\zeta_{2})\right)}\right]}\odot \exp{\left(-\eps
    S_{x}(\zeta_{2})\right)} \\ 
  \xpp &= x_{\bar{m}^{t}} + m^{t}\odot%
    {\left[\xp - \eps{\left(\exp{\left(\eps Q_{x}(\zeta_{3})\right)}\odot \vp%
    + T_{x}(\zeta_{3})\right)}\right]}\odot \exp{\left(-\eps
    S_{x}(\zeta_{3})\right)} \\
  v^{\prime\prime} &= {\left\{\vp +
      \frac{\eps}{2}\left[\partial_{x}\,U(\xpp)\odot%
        \exp{\left(\eps Q_{v}(\zeta_{1})\right)}
  + T_{v}(\zeta_{1})\right]\right\}}\odot 
    \exp{\left(-\frac{\eps}{2}S_{v}(\zeta_{4})\right)}
\end{align}
%
With $\zeta_{1} = (x, \partial_{x}\, U(x), t)$, $\zeta_{2} = (x_{m^{t}}, v,
t)$, $\zeta_{3} = (x^{\prime}_{\bar{m}^{t}}, v, t)$, $\zeta_{4} = (\xpp,
\partial_{x}\, U(\xpp), t)$.
%%%%%%%%%%%%%%%%%%%%%%%%%%%%%%%%%%%%%%%%%%%%%%%%%%%%%%%%%%%%%%%%%%%%%%%%%%%%%%%
%
%%%%%%%%%%%%%%%%%%%%%%%%%%%%%%%%%%%%%%%%%%%%%%%%%%%%%%%%%%%%%%%%%%%%%%%%%%%%%%%
\section{Determinant of the Jacobian}
In terms of the auxiliary functions $S_{i}, Q_{i}, T_{i}$, we can compute the
Jacobian:
%
\begin{align}
  \log|\mathcal{J}| 
  &= \log\bigg|\frac{\partial{\left[\mathbf{FL}_{\theta}\xi\right]}}{\partial
  \xi^{T}}\bigg|\\
  &= d \sum_{t\leq N_{\mathrm{LF}}}
    {\left[\frac{\eps}{2} \mathbbm{1}\cdot S_{v}(\zeta_{1}^{t}) + \eps m^{t}
        \cdot S_{x}(\zeta_{2}^{t}) 
      + \eps \bar{m}^{t} \cdot S_{x}(\zeta_{3}^{t}) + \frac{\eps}{2}\mathbbm{1}
\cdot S_{v}(\zeta_{4}^{t})\right]}.  \end{align}
%
where $N_{\mathrm{LF}}$ is the number of leapfrog steps, and $\zeta_{i}^{t}$
denotes the intermediary variable $\zeta_{i}^{t}$ at time step $t$ and $d$ is
the direction of $\xi$, i.e.\ $d = 1 \,\, (-1)$ for the forward (backward)
update.
%%%%%%%%%%%%%%%%%%%%%%%%%%%%%%%%%%%%%%%%%%%%%%%%%%%%%%%%%%%%%%%%%%%%%%%%%%%%%%%
%
%%%%%%%%%%%%%%%%%%%%%%%%%%%%%%%%%%%%%%%%%%%%%%%%%%%%%%%%%%%%%%%%%%%%%%%%%%%%%%%
